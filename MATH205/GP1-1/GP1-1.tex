\documentclass{ximera}  
\title{Guided Practice 1.1: How do we measure velocity?}  
\begin{document}  
\begin{abstract}  
  Here we look at the problem of measuring velocity.
\end{abstract}  
\maketitle

\section*{Overview}

Our study of calculus begins with a basic building block for
understanding change: the concept of \textbf{average velocity}. Two
formulas for average velocity are introduced in the section and
connected to the concept of the slope of a line. The concept of
\textbf{instantaneous velocity}---central to an understanding of
calculus---is introduced at the end.\par

\section*{Learning objectives}
\subsection*{Basic objectives}

Each student is responsible for gaining proficiency with each of these
tasks \emph{prior} to engaging in class discussions, through the use
of the learning resources (below) and through the working of exercises
(also below).

\begin{itemize}
\item Given a formula for the position of an object as a function of
  time, find the object's position at a specific time.
\item Calculate the average velocity of an object over a given time
  interval using the average velocity formula in the box before
  Activity 1.1.2.
\item Calculate the average velocity of an object over a given time
  interval using the average velocity formula before Example 1.1.2.
\item Explain the differences between the two average velocity
  formulas in this section.
\item Explain the difference between average velocity and
  instantaneous velocity.
\end{itemize}

\subsection*{Advanced objectives}

The following objectives are the subject of class discussion and
further work; they should be mastered by each student \emph{during}
and \emph{following} class discussions.\par

\begin{itemize}
\item Explain the meaning of a negative velocity.
\item Interpret the average velocity of an object geometrically
  through the use of the graph of its position function.
\item Use the second average velocity formula to find the average
  velocity of an object on an interval starting at time $t=a$ and
  ending at time $t=a+h$, where $a$ is given but $h$ is a
  variable. Express the answer in simplest form as a function of $h$.
\item Given the average velocity of an object from time $t=a$ to time $t=a+h$, find its
  instantaneous velocity at the single moment $t=a$. (See Example 1.1.2 and
  Screencast 1.1.3.)
\end{itemize}

\section*{Learning resources}
To gain proficiency in the learning objectives, use the following
resources. You may include other resources if you wish, in addition to
or in replacement of the following.

\subsection*{Reading}
In \emph{Active Calculus}, read Section 1.1. (Stop at the heading
``Exercises.'' Unless I say otherwise, exercises are never
part of a reading assignment.) Make sure to read actively, working
through examples as you go. However, you may skim the activities at
this stage, as we will work through some of them in class.\par

\subsection*{Viewing}

(If you want to review a video again later after completing a Guided
Practice assignment, you can find a link to all the videos by using
the Videos link on Moodle.)

I encourage you to watch all three of the following videos (total
runtime is about 32 minutes) that give additional examples to
supplement the reading. If you're comfortable with using the two
average velocity formulas given in the end-of-section section summary, then
you can probably skip the first two. However, the third video is a
much more in-depth look at an example similar to Example 1.1.2 from the
reading, and you really need to watch that one.

\youtube{6HPe7iwr88k}
\youtube{O_Z9osv6VGk}
\youtube{j8kJubOTkME}

This activity is about creative work.  
\begin{exercise}  
  Choose the best place to work on mathematics.  
  \begin{multipleChoice}  
    \choice{At the library}  
    \choice[correct]{At the caf\'e}  
    \choice{In your office}  
  \end{multipleChoice}  
\end{exercise}

Let's keep going with this.

\begin{exercise}
  Let \(f(x) = x^3\). How about \(f'(x) = \answer{3x^2}\).
\end{exercise}

\end{document}